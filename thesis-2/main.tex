\documentclass{SGGW-thesis}

\INZYNIERSKAtrue
\WZIMtrue

\title{Analiza działania oraz odtworzenie narzędzi front--endowych}
\Etitle{Analysis and reconstruction of front--end tools}
\author{Bartosz Osiński}
\date{2022}
\album{196097}
\thesis{Praca dyplomowa na kierunku:}
\course{Informatyka}
\promotor{dr.\ inż.\ Waldemara Karwowskiego, prof.\ SGGW}
\pworkplace{Instytut Informatyki Technicznej\\Katedra Sztucznej Inteligencji} % TODO

\usepackage{hyperref}

\begin{document}
\maketitle
\statementpage
\abstractpage % TODO napisać streszczenie i słowa kluczowe
{Stworzenie klasy \LaTeX-owej do użytku przy pisaniu pracy dyplomowej w SGGW}
{Tematem niniejszej pracy było zaimplementowanie klasy \LaTeX-owej pozwalającej na formatowanie tekstu zgodnie z wytycznymi nałożonymi przez uczelnię. Praca zawiera dwie
główne części. Pierwsza z nich zawiera opis najważniejszych aspektów implementacji klasy. Natomiast druga część skupia się na sposobie użycia klasy przez osoby piszące prace
dyplomowe.}
{LaTeX, klasa, praca dyplomowa, implementacja, SGGW, Szkoła Główna Gospodarstwa Wiejskiego}
{Creation of the \LaTeX\ Class to be Used When Writing a Thesis at the Warsaw University of Life Sciences -- SGGW}
{The subject of this study was to implement a \LaTeX\ class that allows for text formatting according to the guidelines imposed by the University. The work consists of two
main parts. The first one describes the most important aspects of the implementation. The second part focuses on how to use the class by people writing the theses.}
{LaTeX, class, thesis, implementation, SGGW, Warsaw University of Life Sciences}


{
  % Spis treści może być złożony z pojedynczą interlinią, np. jeśli jedna linia wychodzi na następną stronę.
  % W przeciwnym razie spis treści wstawić bez powyższego rozkazu i klamry.
  \doublespacing
  \tableofcontents
}

\startchapterfromoddpage % niezależnie od długości spisu treści pierwszy rozdział zacznie się na nieparzystej stronie


\chapter{Wstęp}
Współczesne aplikacje internetowe najczęściej składają się z dwóch części: front--endowej oraz back--endowej. Rozróżnienie między nimi polega na tym, gdzie kod jest wykonywany -- jeśli jest on wykonywany po stronie serwera, mówimy o back--endzie, natomiast jeśli obliczenia wykonywane są na komputerze użytkownika, jest to front--end.

Programowanie front--endowe jest jedną z najszybciej rozwijających się dziedzin programowania. Technologie, które pięć lat temu były zdecydowanymi liderami na rynku dziś są uważane za przestarzałe. Tak szybka ewolucja technik developerskich połączona z potrzebą kompatybilności wstecznej aplikacji internetowych (Właściciele aplikacji chcą, aby działały one dla jak największej grupy klientów, a nie tylko dla tych, którzy posiadają najnowsze wersje przeglądarek) generuje problemy, które deweloperzy rozwiązują używając między innymi szerokiego zakresu narzędzi front--endowych.

\section{Charakterystyka narzędzi front-endowych}
Podstawową cechą narzędzi front--endowych -- być może jedyną cechą wspólną dla większości narzędzi -- jest sposób ich konfiguracji. Większość narzędzi jest dostosowana do uruchomienia na dwa sposoby:
\begin{enumerate}
    \item Poprzez \emph{Command Line Interface} -- CLI -- wykonując krótki skrypt w wierszu poleceń takim jak np. Bash
    \item Jako kod, najczęściej w języku Javascript -- narzędzie dostarczane jest developerowi jako biblioteka programistyczna. Programista tworzy plik, który wywołuje funkcje z dostarczonej biblioteki z odpowiednimi argumentami. Następnie za każdym razem, kiedy chce uruchomić narzędzie musi uruchomić stworzony przez siebie plik, np. przy użyciu środowiska Node.js, jeśli kod jest napisany w języku javascript.
\end{enumerate}
Wiele narzędzi pozwala na uruchomienie ich przy użyciu obu tych metod: poprzez CLI w przypadkach, które nie wymagają skomplikowanej konfiguracji ani niestandardowych zachowań, oraz poprzez kod w bardziej skomplikowanych sytuacjach.

Wiele narzędzi łączy obie te metody w taki sposób, że programista tworzy plik, np. javascript, który eksportuje obiekt definiujący konfigurację, a następnie w wierszu poleceń uruchamia narzędzie przekazując mu tylko jeden argument -- ścieżkę do wspomnianego wcześniej pliku.

\section{Wybór typów narzędzi}
Praca ta ma na celu przeanalizowanie działania dwóch typów narzędzi frontendowych:
\begin{itemize}
    \item Module bundler -- Narzędzie, które łączy kod aplikacji rozdzielony na wiele plików w jeden plik, w ten sposób zamieniając kod przyjazny deweloperowi na wydajny kod wspierający starsze przeglądarki, które nie pozwalały na nowoczesne metody importowania i eksportowania symboli w plikach javascript. Przykładami istniejących narzędzi tego typu są \emph{Webpack} oraz \emph{Parcel}
    \item Generator CSS--owych klas atomicznych -- Narzędzie, które przeszukuje kod aplikacji, szukając w niej nazw klas, dla których mogą być zdefiniowane jakieś style CSS, a następnie generuje plik CSS z odpowiednimi stylami. Przykładem może być \emph{Tailwind\cite{Tailwind_jit}} działający w trybie \emph{Just in Time} oraz \emph{UnoCSS\cite{unocss}}
\end{itemize}
Oba te typy narzędzi posiadają swoich przedstawicieli, na których będzie można się skupić podczas analizowania sposobu ich działania.


\chapter{Module bundler}
\emph{Module budler} to -- jak wspomniano wyżej -- narzędzie służące do "sklejenia" kodu znajdującego się w wielu plikach w jeden duży plik. Służy to kilku celom:
\begin{itemize}
    \item Zmniejszenie ilości zapytań do serwera -- Duże, komercyjne aplikacje nierzadko składają się z setek lub tysięcy plików. Wdrożenie aplikacji rozdzielonej w taki sposób wiązało by się z wysyłaniem setek lub tysięcy zapytań z przeglądarki na serwer podczas każdorazowego odwiedzenia strony. Byłoby to duże obciążenie dla przeglądarki oraz ogromne obciążenie dla serwera.
    \item Utrzymanie kompatybilności wstecznej -- 
\end{itemize}

\section{Sposoby bundlowania}
\section{Transformacja kodu javascript przy użyciu parsera}
\section{Zależności cykliczne}
\section{Testowanie module bundlera}

\chapter{Generator CSSowych klas atomicznych}
\section{Czym są klasy atomiczne?}
\section{Powody używania generatora klas atomicznych}
\section{Sposób działania}

\chapter{Przykładowa aplikacja stworzona przy użyciu zaprezentowanych narzędzi}

\chapter{Podsumowanie i wnioski}


\begin{thebibliography}{9}
% \bibitem{wymagania}
% \textit{Zarządzenie nr 34 Rektora Szkoły Głównej Gospodarstwa Wiejskiego w Warszawie z dnia 01 czerwca 2016 r.\ w~ sprawie wprowadzenia ,,Wytycznych dotyczących
% przygotowywania prac dyplomowych w~Szkole Głównej Gospodarstwa Wiejskiego w Warszawie''}, Załączniki 1 i~2
% \url{https://www.sggw.pl/dla-studentow/informacje-formalno-prawne/dokumenty-do-pobrania}
% $\rightarrow$ Praca dyplomowa (dostęp: 04.01.2017)
% \bibitem{latexclass}
% \textit{\LaTeXe\ for class and package writers} \url{http://www.latex-project.org/help/documentation/} $\rightarrow$ LaTeX2e for class... (dostęp 04.01.2017)
% \bibitem{latexcompanion}
% Frank Mittelbach and Michel Goossens with Johannes Braams, David Carlisle and Chris Rowley,
% \textit{The \LaTeX\ Companion}. Second Edition,
% Addison-Wesley, 2004.
% \bibitem{tds}
% \textit{A Directory Structure for \TeX\ Files} \url{http://tug.org/tds/tds.html} (dostęp 11.01.2017)
\bibitem{Tailwind_jit}
\textit{Just-in-Time Mode -- Tailwind CSS}
\url{https://v2.tailwindcss.com/docs/just-in-time-mode} (dostęp 28.02.2022)

\bibitem{unocss}
\textit{Repozytorium projektu UnoCSS w serwisie Github}
\url{https://github.com/unocss/unocss} (dostęp 28.02.2022)

\bibitem{Exploring_JS}
Axel Rauschmayer, \textit{Exploring JS}

\end{thebibliography}

\beforelastpage

\end{document}
